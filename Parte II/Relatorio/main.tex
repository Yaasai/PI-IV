% !TeX spellcheck = pt_BR
\documentclass[12pt]{article}
\usepackage[margin=0.5in]{geometry}
\setlength{\parindent}{0pt} 
\setlength{\parskip}{5pt} 
\pagenumbering{gobble}

\usepackage[utf8]{inputenc}
\usepackage{amsmath,amsthm,amssymb}
\usepackage{graphicx}
\usepackage{float}
\usepackage[portuguese]{babel}


\title{PI IV - Visualização de dados numéricos}
\author{Lucas Gomes Santana}
\date{\today}

\begin{document}

\maketitle

Para as visualizações foi utilizado um subconjunto dos dados da GISS Surface Temperature Analysis (GISTEMP v4)\cite{gistemp}, contendo números das alterações da média de temperatura global, do hemisfério sul e norte. Em ambos gráficos o eixo y representa a alteração da média de temperatura anual enquanto o eixo x representa o ano da alteração. \\

\textbf{\Large Gráfico de linhas}\\
A primeira representação dos dados numéricos é por meio de linhas.  Como os dados representam as alterações com o passar do tempo, um gráfico do tipo linha pode mostrar a progressão e tendências futuras.
\setcounter{figure}{0}
\begin{figure}[H]
\centering
\includegraphics[width = 0.714\textwidth]{lines.png}
\label{fig:A.1}
\caption{Gráfico de linha mostrando a alteração de temperatura anual da terra pelo tempo.}
\end{figure}

\textbf{\Large Boxplot}\\
A segunda representação é do tipo 'boxplot'. Esse tipo de gráfico é usado para mostrar de forma rápida a distribuição dos dados, outliers e o quão simétricos eles são.
\setcounter{figure}{1}
\begin{figure}[H]
	\centering
	\includegraphics[width = 0.75\textwidth]{boxplot.png}
	\label{fig:A.2}
	\caption{Gráfico de linha mostrando a alteração de temperatura anual da terra pelo tempo.}
\end{figure}

\textbf{\Large Comparação}
\begin{center}
	\begin{tabular}{ |c|c|c| } 
		\hline
		 	  & Linhas & Boxplot \\ 
		\hline
		Tendência &  &  \\ 
		\hline
		Progressão &  &  \\ 
		\hline
		Distribuição &  &  \\ 
		\hline
		Simetria &  &  \\ 
		\hline
	\end{tabular}
\end{center}
\newpage
\begin{thebibliography}{9}
	\bibitem{gistemp} 
	GISTEMP Team, 2019: GISS Surface Temperature Analysis (GISTEMP), version 4. NASA Goddard \\
	Institute for Space Studies. Dataset accessed 2019-09-17 at \\\texttt{https://data.giss.nasa.gov/gistemp/}.
	
	\bibitem{latexcompanion} 
	Lenssen, N., G. Schmidt, J. Hansen, M. Menne,A. Persin,R. Ruedy, and D. Zyss, 2019 
	\textit{Improvements in the GISTEMP uncertainty model. J. Geophys. Res. Atmos., 124, no. 12, 6307-6326, doi:10.1029/2018JD029522.}.
\end{thebibliography}

\end{document}